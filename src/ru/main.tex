\documentclass[a4paper, 11pt]{article}
\usepackage [utf8] {inputenc}
% Russian and english word-wrap
\usepackage[english,russian]{babel}
% Start new lines with a shift from the left
\usepackage{indentfirst}
% Choosing of internal TEX-encoding
\usepackage[T2A]{fontenc}

\usepackage{cmap}
\usepackage{xcolor}
\usepackage{longtable}

\usepackage{hyperref}
\definecolor{LINKCOLOUR}{rgb}{0.1,0.0,0.9}
\hypersetup{colorlinks,breaklinks,urlcolor=LINKCOLOUR,linkcolor=LINKCOLOUR}

\usepackage{geometry} % Page margins
\geometry{left=2.5cm}
\geometry{right=2cm}
\geometry{top=2cm}
\geometry{bottom=2cm}

\usepackage{titlesec} % Used to customize the \section command
\titleformat{\section}{\Large\raggedright}{\color{green}}{0em}{}[\titlerule] % Text formatting of sections
\titlespacing{\section}{2pt}{3pt}{3pt} % Spacing around sections

\pagestyle{empty}

\begin{document}

{\LARGE\textbf{Сергей Никитин}}

\begin{flushright}
    {\itshape
        ул. Полярная, 12, г. Москва, Россия                                                     \\
        +7-968-708-71-71                                                                        \\
        \href{mailto:snikitin@outlook.com}{snikitin@outlook.com}                                \\
        \href{https://github.com/SergNikitin}{github.com/SergNikitin}                           \\
        \href{http://ru.linkedin.com/in/nikitinsergey}{ru.linkedin.com/in/nikitinsergey}        \\
        \href{http://stackoverflow.com/users/3375765/sergey-nikitin}{stackoverflow.com profile} \\
    }
\end{flushright}

\section{Навыки}
\begin{tabular}{ll}
    Языки программирования   & C++ (C++14), C (C99), Lua (5.1, 5.2)            \\
    Прикладные программы     & VS2015, GNU-Make, CMake, Doxygen, LaTeX, Matlab \\
    Системы контроля версий  & Git, Subversion                                 \\
    3D-моделирование         & Autodesk Inventor
\end{tabular}

\section{Опыт работы}
\begin{longtable}{p{25mm}|p{110mm}}
    Ноябрь 2014 -       & \textbf{BitBox Ltd.}                                                        \\
    по настоящее время  & C/C++ разработчик - игра \href{http://lifeisfeudal.com/mmo}{Life Is Feudal} \\
                        &
    \begin{itemize}
        \item   Разработал с нуля систему динамической генерации геометрии
                игровых туннелей и шахт (на основе алгоритма
                \href{https://ru.wikipedia.org/wiki/Marching_cubes}{марширующих кубов})
                - хорошая алгоритмическая и архитектурная задача
        \item   Переработал систему мульти-поточной загрузки/выгрузки игровых
                объектов в зависимости от расстояния от игрока до объекта
        \item   Внедрил в игровой движок поддержку Unicode на основе библиотеки
                \href{http://site.icu-project.org/}{ICU} (включая внутренние
                классы и функции для работы со строками) и локализации на
                различные языки
        \item   Реализовал ряд игровых механик
        \item   Непрерывно занимаюсь постепенным переводом кода игрового движка
                и игровой логики с устаревших С++-практик на более удобные и
                безопасные (move-семантика, чёткое владение ресурсами через
                smart-pointer-ы и т.д.)
        \item   Провожу большое количество ревью кода других членов команды
    \end{itemize}                                                                   \\

    \pagebreak[4]
    Апрель 2013 -       & \textbf{«Лаборатория Трёхмерного зрения»}                 \\
    Ноябрь 2014         & Младший C/C++ разработчик                                 \\
                        &
    \begin{itemize}
        \item   Развивал системы автоматического управления семейства бытовых
                роботов R.BOT Synergy - интересный опыт практического применения
                фундаментальных знаний, полученных в университете
        \item   Разрабатывал пакет комплексного тестирования и отладки аппаратной
                части R.BOT Synergy с использованием графической библиотеки
                Tecgraf IUP - опыт плотного взаимодействия с пользователями,
                создания графических интерфейсов
        \item   Поддерживал ПО на платформу x86 (Windows, Linux), написанное на
                C++, Lua 5.1 для управления, отладки и контроля за
                микроконтролерными системами
        \item   Развивал ПО для корпоративной системы контроля и управления
                доступом
    \end{itemize}                                                                   \\

    \pagebreak[3]
    Июль 2011 - & \textbf{«Лаборатория трехмерного зрения»}                         \\
    Апрель 2013 & Техник                                                            \\
                &
    \begin{itemize}
        \item   Комплексная диагностика бытовых роботов
        \item   Производство бытовых роботов
    \end{itemize}                                                                   \\
\end{longtable}

\section{Образование}
\begin{tabular}{p{25mm}|p{110mm}}
2008 - 2014         & «МГТУ им Н.Э. Баумана»                            \\
                    & Факультет «Специальное машиностроение»            \\
                    & Кафедра «Специальная робототехника и мехатроника» \\
                    & URL: \href{http://bmstu.ru}{bmstu.ru}             \\
                    & Cтепень: Специалист                               \\
                    & Средний балл: 4,95/5                              \\
                    & Диплом с отличием
\end{tabular}

\section{Научные и фундаментальные знания}
\begin{itemize}
    \item Системы автоматического управления
    \item Основы работы микропроцессорной техники
    \item Линейная алгебра, дискретная математика
    \item Теоретическая механика
\end{itemize}

\section{Иностранные языки}
\begin{itemize}
    \item   Русский    - родной язык
    \item   Английский - advanced, непрерывно поддерживаю и развиваю знания через
            книги, фильмы и игры на английском
\end{itemize}

\section{Профессиональные интересы}
\begin{itemize}
    \item   Разработка видео-игр
    \item   Работа над модулями/фреймворками/утилитами, упрощающими написание
            кода себе и другим разработчикам, делающими код более выразительным
            и безопасным
    \item   Продумывание программных интерфейсов, архитектуры программных модулей
    \item   Расширение знаний через профессиональную литературу и видео-лекции
            (записи с CppCon и т.д.). Последняя прочтённая книга - «Modern
            Effective C++», Scott Meyers. Сейчас читаю «C++ Concurrency in
            Action», Anthony Williams.
    \item   3D-моделирование
\end{itemize}

\section{Прочие интересы}
\begin{itemize}
    \item   Художественная литература -
            \href{https://www.goodreads.com/user/show/29629010-sergey-nikitin}{Goodreads.com profile}
    \item   Музыкальные концерты U2, Depeche Mode, Coldplay, Rammstein
    \item   Спорт: скалолазание, плавание
\end{itemize}

\end{document}
