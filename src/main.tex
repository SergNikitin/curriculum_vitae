\documentclass[a4paper, 11pt]{article}
\usepackage [utf8] {inputenc}
% включаем переносы для русского и английского языка
\usepackage[english,russian]{babel}
% Начинать первый параграф раздела следует с красной строки
\usepackage{indentfirst}
% Выбор внутренней TEX−кодировки
\usepackage[T2A]{fontenc}

\usepackage{cmap}
\usepackage{xcolor}

\usepackage{hyperref}
\definecolor{LINKCOLOUR}{rgb}{0.1,0.0,0.9}
\hypersetup{colorlinks,breaklinks,urlcolor=LINKCOLOUR,linkcolor=LINKCOLOUR}

\usepackage{geometry} % Поля страницы
\geometry{left=2.5cm}
\geometry{right=2cm}
\geometry{top=2cm}
\geometry{bottom=2cm}

\usepackage{titlesec} % Used to customize the \section command
\titleformat{\section}{\Large\raggedright}{\color{green}}{0em}{}[\titlerule] % Text formatting of sections
\titlespacing{\section}{2pt}{3pt}{3pt} % Spacing around sections

\pagestyle{empty}

\begin{document}

{\LARGE\textbf{Сергей Никитин}}

\begin{flushright}
    {\itshape
        ул. Полярная, 12, г. Москва, Россия                                                 \\
        +7-968-708-71-71                                                                    \\
        \href{mailto:snikitin@outlook.com}{snikitin@outlook.com}                            \\
        \href{https://github.com/SergNikitin}{github.com/SergNikitin}                       \\
        \href{http://ru.linkedin.com/in/nikitinsergey}{ru.linkedin.com/in/nikitinsergey}    \\
    }
\end{flushright}

\section{Навыки}
\begin{tabular}{ll}
    Языки программирования   & C (C99), C++ (ISO/IEC 2003), Lua (5.1, 5.2)  \\
    Прикладные программы     & GNU-Make, CMake, Doxygen, LaTeX, Matlab      \\
    Системы контроля версий  & Git, Subversion                              \\
    3D-моделирование         & Autodesk Inventor, Blender
\end{tabular}

\section{Опыт работы}
\begin{tabular}{p{25mm}|p{110mm}}
    Апрель 2013 -       & \textbf{«Лаборатория трехмерного зрения»}                 \\
    по настоящее время  & Младший C/C++ разработчик                                 \\
                        &
    \begin{itemize}
        \item   Развиваю системы автоматического управления семейства бытовых
                роботов R.BOT Synergy - интересный опыт практического применения
                фундаментальных знаний, полученных в университете
        \item   Разрабатываю пакет комплексного тестирования и отладки аппаратной
                части R.BOT Synergy с использованием графической библиотеки
                Tecgraf IUP - опыт плотного взаимодействия с пользователями,
                создания графических интерфейсов
        \item   Поддерживаю ПО на платформу x86, написанное на C++, Lua 5.1
                для управления, отладки и контроля за микроконтролерными системами
        \item   Развиваю ПО для корпоративной системы контроля и управления
                доступом
    \end{itemize}                                                                   \\

    Июль 2011 - & \textbf{«Лаборатория трехмерного зрения»}                         \\
    Апрель 2013 & Техник                                                            \\
                &
    \begin{itemize}
        \item   Комплексная диагностика бытовых роботов
        \item   Производство бытовых роботов
    \end{itemize}                                                                   \\
\end{tabular}

\section{Образование}
\begin{tabular}{p{25mm}|p{110mm}}
2008 - 2014         & «МГТУ им Н.Э. Баумана»                            \\
                    & Факультет «Специальное машиностроение»            \\
                    & Кафедра «Специальная робототехника и мехатроника» \\
                    & URL: \href{http://bmstu.ru}{bmstu.ru}             \\
                    & Cтепень: Специалист                               \\
                    & Средний балл: 4,95                                \\
                    & Диплом с отличием
\end{tabular}

\section{Научные и фундаментальные знания}
\begin{itemize}
    \item Системы автоматического управления
    \item Основы работы микропроцессорной техники
    \item Линейная алгебра, дискретная математика
    \item Теоретическая механика
\end{itemize}

\section{Иностранные языки}
\begin{itemize}
    \item   Русский    - родной язык
    \item   Английский - advanced, непрерывно поддерживаю и развиваю знания через
            книги, фильмы и игры на английском
\end{itemize}

\section{Профессиональные интересы}
\begin{itemize}
    \item   Классические алгоритмы и их реализация, разбираю книгу Т. Кормэна,
            ``Introduction to Algorithms, Third Edition''
    \item   Изучение путей общего улучшения собственного (а иногда и чужого)
            кода посредством изучения таких книг как «Чистый код» Р. Мартина и
            «Эффективный C++» С. Мейерса
    \item   Изучение Python
    \item   Лекции и семинары по программированию, последнее посещённое мероприятие
            - лекция С. Мейерса в рамках ``C++ Party'' в Яндексе на тему развития
            С++
    \item   3D-моделирование
\end{itemize}

\section{Прочие интересы}
\begin{itemize}
    \item   Разработка игр: в компании с моим другом разрабатываем
            небольшую игру, в которой используем C++, Polycode Framework и CMake -
            \href{https://github.com/thegreenbox/helium}{github.com/TheGreenBox/Helium}
    \item   Художественная литература -
            \href{https://www.goodreads.com/user/show/29629010-sergey-nikitin}{Goodreads.com profile}
    \item   Музыкальные концерты U2, Depeche Mode, Coldplay, Rammstein
    \item   Спорт: бально-спортивные танцы, плавание
\end{itemize}

\end{document}
